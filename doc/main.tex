% generated by GAPDoc2LaTeX from XML source (Frank Luebeck)
\documentclass[a4paper,11pt]{report}

\usepackage{a4wide}
\sloppy
\pagestyle{myheadings}
\usepackage{amssymb}
\usepackage[latin1]{inputenc}
\usepackage{makeidx}
\makeindex
\usepackage{color}
\definecolor{FireBrick}{rgb}{0.5812,0.0074,0.0083}
\definecolor{RoyalBlue}{rgb}{0.0236,0.0894,0.6179}
\definecolor{RoyalGreen}{rgb}{0.0236,0.6179,0.0894}
\definecolor{RoyalRed}{rgb}{0.6179,0.0236,0.0894}
\definecolor{LightBlue}{rgb}{0.8544,0.9511,1.0000}
\definecolor{Black}{rgb}{0.0,0.0,0.0}

\definecolor{linkColor}{rgb}{0.0,0.0,0.554}
\definecolor{citeColor}{rgb}{0.0,0.0,0.554}
\definecolor{fileColor}{rgb}{0.0,0.0,0.554}
\definecolor{urlColor}{rgb}{0.0,0.0,0.554}
\definecolor{promptColor}{rgb}{0.0,0.0,0.589}
\definecolor{brkpromptColor}{rgb}{0.589,0.0,0.0}
\definecolor{gapinputColor}{rgb}{0.589,0.0,0.0}
\definecolor{gapoutputColor}{rgb}{0.0,0.0,0.0}

%%  for a long time these were red and blue by default,
%%  now black, but keep variables to overwrite
\definecolor{FuncColor}{rgb}{0.0,0.0,0.0}
%% strange name because of pdflatex bug:
\definecolor{Chapter }{rgb}{0.0,0.0,0.0}
\definecolor{DarkOlive}{rgb}{0.1047,0.2412,0.0064}


\usepackage{fancyvrb}

\usepackage{mathptmx,helvet}
\usepackage[T1]{fontenc}
\usepackage{textcomp}


\usepackage[
            pdftex=true,
            bookmarks=true,        
            a4paper=true,
            pdftitle={Written with GAPDoc},
            pdfcreator={LaTeX with hyperref package / GAPDoc},
            colorlinks=true,
            backref=page,
            breaklinks=true,
            linkcolor=linkColor,
            citecolor=citeColor,
            filecolor=fileColor,
            urlcolor=urlColor,
            pdfpagemode={UseNone}, 
           ]{hyperref}

\newcommand{\maintitlesize}{\fontsize{50}{55}\selectfont}

% write page numbers to a .pnr log file for online help
\newwrite\pagenrlog
\immediate\openout\pagenrlog =\jobname.pnr
\immediate\write\pagenrlog{PAGENRS := [}
\newcommand{\logpage}[1]{\protect\write\pagenrlog{#1, \thepage,}}
%% were never documented, give conflicts with some additional packages

\newcommand{\GAP}{\textsf{GAP}}

%% nicer description environments, allows long labels
\usepackage{enumitem}
\setdescription{style=nextline}

%% depth of toc
\setcounter{tocdepth}{1}





%% command for ColorPrompt style examples
\newcommand{\gapprompt}[1]{\color{promptColor}{\bfseries #1}}
\newcommand{\gapbrkprompt}[1]{\color{brkpromptColor}{\bfseries #1}}
\newcommand{\gapinput}[1]{\color{gapinputColor}{#1}}


\begin{document}

\logpage{[ 0, 0, 0 ]}
\begin{titlepage}
\mbox{}\vfill

\begin{center}{\maintitlesize \textbf{\textsf{Ferret}\mbox{}}}\\
\vfill

\hypersetup{pdftitle=\textsf{Ferret}}
\markright{\scriptsize \mbox{}\hfill \textsf{Ferret} \hfill\mbox{}}
{\Huge \textbf{Backtrack Search in Permutation Groups\mbox{}}}\\
\vfill

{\Huge Version 0.4.0\mbox{}}\\[1cm]
{3 November 2014\mbox{}}\\[1cm]
\mbox{}\\[2cm]
{\Large \textbf{ Christopher Jefferson   \mbox{}}}\\
\hypersetup{pdfauthor= Christopher Jefferson   }
\end{center}\vfill

\mbox{}\\
{\mbox{}\\
\small \noindent \textbf{ Christopher Jefferson   }  Email: \href{mailto://caj21@st-andrews.ac.uk} {\texttt{caj21@st-andrews.ac.uk}}\\
  Homepage: \href{http://caj.host.cs.st-andrews.ac.uk/} {\texttt{http://caj.host.cs.st-andrews.ac.uk/}}}\\
\end{titlepage}

\newpage\setcounter{page}{2}
{\small 
\section*{Copyright}
\logpage{[ 0, 0, 1 ]}
 \index{License} {\copyright} 2013-2014 by Christopher Jefferson and Steve Linton

 \textsf{Ferret} package is free software; you can redistribute it and/or modify it under the
terms of the \href{http://www.fsf.org/licenses/gpl.html} {GNU General Public License} as published by the Free Software Foundation; either version 2 of the License,
or (at your option) any later version. \mbox{}}\\[1cm]
\newpage

\def\contentsname{Contents\logpage{[ 0, 0, 2 ]}}

\tableofcontents
\newpage

     
\chapter{\textcolor{Chapter }{The Ferret Package}}\label{The Ferret Package}
\logpage{[ 1, 0, 0 ]}
\hyperdef{L}{X84F30BD780680D41}{}
{
  \index{Ferret package} This chapter describes the \textsf{GAP} package Ferret. Ferret implements highly efficient implementations of a range
of search algorithms on permutation groups. These algorithms fall into two
broad categories: 
\begin{itemize}
\item  \emph{Algorithms which find groups and cosets with particular properties}. This include graph automorphisms, stabilizers and group intersection. 
\item  \emph{Canonical Images}. This involves finding the canonical image of an object (graph, set, list,
transformation and others) in a given group of coset. 
\end{itemize}
 If you are interested in if Ferret can be applied to another problem, please
contact the authors, who will be happy to look into if your problem can be
solved with Ferret.  
\section{\textcolor{Chapter }{Replacing Built-in functionality}}\label{Replacing Built-in functionality}
\logpage{[ 1, 1, 0 ]}
\hyperdef{L}{X862CDAE07EF5953E}{}
{
  Ferret automatically installs methods which replace GAP's a number of GAP's
built-in functionality: 
\begin{itemize}
\item  \emph{Intersection} for a list of permutation groups. 
\item  \emph{Stabilizer(G,S,Action)} for a permutation group G, and the actions: 
\begin{itemize}
\item OnSets
\item OnOnSets
\item OnSetsDisjointSets
\item OnSetsSets
\item OnTuples
\item OnPairs
\item OnDirectedGraph
\end{itemize}
 
\item  \emph{Stabilizer(G, S)} for a permutation group G and a: 
\begin{itemize}
\item permutation
\item transformation
\item partial permutation
\end{itemize}
 
\end{itemize}
 If you would like to disable this functionality, you can use \ref{EnableFerretOverloads}. 

\subsection{\textcolor{Chapter }{EnableFerretOverloads}}
\logpage{[ 1, 1, 1 ]}\nobreak
\hyperdef{L}{X7E455E297809B021}{}
{\noindent\textcolor{FuncColor}{$\triangleright$\ \ \texttt{EnableFerretOverloads({\mdseries\slshape [active]})\index{EnableFerretOverloads@\texttt{EnableFerretOverloads}}
\label{EnableFerretOverloads}
}\hfill{\scriptsize (function)}}\\


 if \mbox{\texttt{\mdseries\slshape active}} (a bool) is true, then enable Ferret specialisations of Intersection and
Stabilizer. Call with \mbox{\texttt{\mdseries\slshape active}} false to disable. }

 }

 }

        
\chapter{\textcolor{Chapter }{The Solve Method}}\label{SolveChapter}
\logpage{[ 2, 0, 0 ]}
\hyperdef{L}{X83B607B385FBB68A}{}
{
  The central functionality of the Ferret package is based around the Solve
method. This function performs a backtrack search, using the permutation
backtracking algorithm, over a set of groups or cosets. Often users will want
to use a higher level function which wraps this functionality, such as
Stabilizer or Intersection. The solve function accepts a list of groups or
cosets, and finds their intersection. For efficiency reasons, these groups can
be specified in a variety of different ways. As an example, we will consider
how to implement Stabilizer(G, S, OnSets), the stabilizer of a set S in a
permutation group G using Solve (this is not necessary, as when Ferret is
loaded this method is replaced with a Ferret-based implementation). Another
way of viewing Stabilizer(G, S, OnSets) is as the Intersection of G with the
Stabilizer of (Sym(n), S, OnSets), where Sym(n) is the symmetric group on n
points, and n is at least as large as the largest moved point in G. Solve
takes a list of objects which represent groups. Two of these are
ConInGroup(G), which represents the group G, and ConStabilize(S, OnSets),
which represents the group which stabilizes S. We find the intersection of
these two groups by Solve([ConInGroup(G), ConStabilize(S, OnSets)]). 
\section{\textcolor{Chapter }{Methods of representing groups in Ferret}}\label{Representing groups in Ferret}
\logpage{[ 2, 1, 0 ]}
\hyperdef{L}{X7ADF81FD7F1709BB}{}
{
  Groups and cosets must be represented in a way which Ferret can understand.
The following list gives all the types of groups and cosets which Ferret
accepts, and how to construct them. 
\subsection{\textcolor{Chapter }{ConStabilize}}\logpage{[ 2, 1, 1 ]}
\hyperdef{L}{X7C1B860E87BD036C}{}
{
\noindent\textcolor{FuncColor}{$\triangleright$\ \ \texttt{ConStabilize({\mdseries\slshape object, action})\index{ConStabilize@\texttt{ConStabilize}!for an object and an action}
\label{ConStabilize:for an object and an action}
}\hfill{\scriptsize (function)}}\\
\noindent\textcolor{FuncColor}{$\triangleright$\ \ \texttt{ConStabilize({\mdseries\slshape object, n})\index{ConStabilize@\texttt{ConStabilize}!for a transformation or partial perm}
\label{ConStabilize:for a transformation or partial perm}
}\hfill{\scriptsize (function)}}\\


 In the first form this represents the group which stabilises \mbox{\texttt{\mdseries\slshape object}} under \mbox{\texttt{\mdseries\slshape action}}. The currently allowed actions are OnSets, OnSetsSets, OnSetsDisjointSets,
OnTuples, OnPairs and OnDirectedGraph. In the second form it represents the
stabilizer of a partial perm or transformation in the symmetric group on \mbox{\texttt{\mdseries\slshape n}} points. Both of these methods are for constructing arguments for the \texttt{Solve} (\ref{Solve}) method. }

 

\subsection{\textcolor{Chapter }{ConInGroup}}
\logpage{[ 2, 1, 2 ]}\nobreak
\hyperdef{L}{X7EE49BAB82A1BB20}{}
{\noindent\textcolor{FuncColor}{$\triangleright$\ \ \texttt{ConInGroup({\mdseries\slshape G})\index{ConInGroup@\texttt{ConInGroup}}
\label{ConInGroup}
}\hfill{\scriptsize (function)}}\\


 Represents the permutation group \mbox{\texttt{\mdseries\slshape G}}, as an argument for \texttt{Solve} (\ref{Solve}). }

 These methods are both used with Solve: 

\subsection{\textcolor{Chapter }{Solve}}
\logpage{[ 2, 1, 3 ]}\nobreak
\hyperdef{L}{X7A415C2480970A43}{}
{\noindent\textcolor{FuncColor}{$\triangleright$\ \ \texttt{Solve({\mdseries\slshape constraints[, rec]})\index{Solve@\texttt{Solve}}
\label{Solve}
}\hfill{\scriptsize (function)}}\\


 Finds the intersection of the list \mbox{\texttt{\mdseries\slshape constraints}}. Each member of \mbox{\texttt{\mdseries\slshape constraints}} should be a group or coset generated by one of \texttt{ConInGroup} (\ref{ConInGroup}) or ConStabilize. The optional second argument allows configuration options to
be passed in. }

 }

 }

        
\chapter{\textcolor{Chapter }{Installing and Loading the Ferret Package}}\label{Installing and Loading the Ferret Package}
\logpage{[ 3, 0, 0 ]}
\hyperdef{L}{X782C0DD47E1D28EF}{}
{
   
\section{\textcolor{Chapter }{Unpacking the Ferret Package}}\label{Unpacking the Ferret Package}
\logpage{[ 3, 1, 0 ]}
\hyperdef{L}{X8527DE187ADA7D7A}{}
{
  If the Ferret package was obtained as a part of the \textsf{GAP} distribution from the ``Download'' section of the \textsf{GAP} website, you may proceed to Section \ref{Compiling Binaries of the Ferret Package}. Alternatively, the Ferret package may be installed using a separate archive,
for example, for an update or an installation in a non-default location (see  (\textbf{Reference: GAP Root Directories})). 

 Below we describe the installation procedure for the \texttt{.tar.gz} archive format. Installation using other archive formats is performed in a
similar way. 

 It may be unpacked in one of the following locations: 
\begin{itemize}
\item  in the \texttt{pkg} directory of your \textsf{GAP}{\nobreakspace}4 installation; 
\item  or in a directory named \texttt{.gap/pkg} in your home directory (to be added to the \textsf{GAP} root directory unless \textsf{GAP} is started with \texttt{-r} option); 
\item  or in a directory named \texttt{pkg} in another directory of your choice (e.g.{\nobreakspace}in the directory \texttt{mygap} in your home directory). 
\end{itemize}
 In the latter case one one must start \textsf{GAP} with the \texttt{-l} option, e.g.{\nobreakspace}if your private \texttt{pkg} directory is a subdirectory of \texttt{mygap} in your home directory you might type: 

 {\nobreakspace}{\nobreakspace}\texttt{gap -l ";\mbox{\texttt{\mdseries\slshape myhomedir}}/mygap"} 

 where \mbox{\texttt{\mdseries\slshape myhomedir}} is the path to your home directory, which (since \textsf{GAP}{\nobreakspace}4.3) may be replaced by a tilde (the empty path before the
semicolon is filled in by the default path of the \textsf{GAP}{\nobreakspace}4 home directory). }

  
\section{\textcolor{Chapter }{Compiling Binaries of the Ferret Package}}\label{Compiling Binaries of the Ferret Package}
\logpage{[ 3, 2, 0 ]}
\hyperdef{L}{X7DB615628530240D}{}
{
  After unpacking the archive, go to the newly created \texttt{example} directory and call \texttt{./configure} to use the default \texttt{../..} path to the \textsf{GAP} home directory or \texttt{./configure \mbox{\texttt{\mdseries\slshape path}}} where \mbox{\texttt{\mdseries\slshape path}} is the path to the \textsf{GAP} home directory, if the package is being installed in a non-default location.
So for example if you install the package in the \texttt{\texttt{\symbol{126}}/.gap/pkg} directory and the \textsf{GAP} home directory is \texttt{\texttt{\symbol{126}}/gap4r5} then you have to call 

 
\begin{Verbatim}[commandchars=!@|,fontsize=\small,frame=single,label=Example]
  ./configure ../../../gap4r5/
\end{Verbatim}
 

 This will fetch the architecture type for which \textsf{GAP} has been compiled last and create a \texttt{Makefile}. Now simply call 

 
\begin{Verbatim}[commandchars=!@|,fontsize=\small,frame=single,label=Example]
  make
\end{Verbatim}
 

 to compile the binary and to install it in the appropriate place. }

  
\section{\textcolor{Chapter }{Loading the Ferret Package}}\label{Loading the Ferret Package}
\logpage{[ 3, 3, 0 ]}
\hyperdef{L}{X7BA03640834E607E}{}
{
  To use the Ferret Package you have to request it explicitly. This is done by
calling \texttt{LoadPackage} (\textbf{Reference: LoadPackage}): 

 
\begin{Verbatim}[commandchars=!@|,fontsize=\small,frame=single,label=Example]
  !gapprompt@gap>| !gapinput@LoadPackage("example");|
  true
\end{Verbatim}
 

 If you want to load the Ferret package by default, you can put the \texttt{LoadPackage} command into your \texttt{gaprc} file (see Section{\nobreakspace} (\textbf{Reference: The gap.ini and gaprc files})). }

 }

    \def\indexname{Index\logpage{[ "Ind", 0, 0 ]}
\hyperdef{L}{X83A0356F839C696F}{}
}

\cleardoublepage
\phantomsection
\addcontentsline{toc}{chapter}{Index}


\printindex

\newpage
\immediate\write\pagenrlog{["End"], \arabic{page}];}
\immediate\closeout\pagenrlog
\end{document}
